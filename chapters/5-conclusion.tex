\chapter{結論}
\label{chapter:conclusion}


本篇論文提出並實作 \LHCN 這個建立在網路功能虛擬化下的低延遲高流量 5G 核心網路平臺,透過 OpenNetVM 平臺的移植與對同一節點佈署的觀察,在不改變 3GPP 定義的 5G 架構與溝通流程下,提供了包含控制端與用戶端的效能都比以往 3GPP 框架下基於 Berkeley Socket 的 5G 核心網路高的 5G 核心網路框架。
在控制端,我們利用共享記憶體的加速,實踐於 SBI 與 N4 界面上,獲得了比傳統架構低一半的傳遞延遲。
在用戶端上,透過利用 DPDK 越過核心網路疊的效能損耗以及 TSS 的規則加速查找,我們可以比傳統架構多 11 倍的吞吐量。
最後,\LHCN 提出了更有效率的換手流程,利用 UPF 緩衝取代 3GPP 架構中提供的封包間接轉發機制,避免了過長的封包轉發路徑。
