\chapter{實驗與效能評估}
\label{chapter:evaluation}

為驗證本論文設計之方法,將 free5GC 核心網路移植至 OpenNetVM,使用其所提供之 ABI 能有效提升核心網路之效能,本實驗分別對原生之 free5GC 核心網路部屬於 Ubuntu Linux,與經過 OpenNetVM 移植過後之 free5GC 核心網路一樣部屬於 Ubuntu Linux,分別對其控制端(control plane)與用戶端(user plane)之效能進行測試,最終比對其結果,以驗證經過 OpenNetVM 移植過後的 free5GC 核心網路確實能有效提供更好的性能。

\section{實驗環境設定}
\label{sec:evaluation_env}

我們所使用的硬體與軟體規格如下:
\begin{itemize}
\item 中央處理器: Intel(R) Core(TM) i7-9700 CPU @ 3.00GHz
\item 處理器核心數: 8
\item 網卡: Intel Corporation Ethernet Controller X710/X557-AT 10GBASE-T 1589
\item 作業系統: centos-release-7-8.2003.0.el7.centos.x86\_64
\item 作業系統核心版本: 3.10.0-1127.el7.x86\_64
\end{itemize}

我們的實驗在邏輯上(logically)將系統分為三個部件(component),邊緣端(edge)、核心網路(core network)、與數據網路(data network)。邊緣端在真實網路下即是手機、基地臺與相關部件,以下會以 UE-RAN simulator 代稱。核心網路如第二章節所講解的,是第五代行動通訊網路中,放在機房(data center)中,負責統籌處理控制端(control plane)與轉發(forward)、封裝(encapsulate)、解封裝(decapsulate)使用者端(user plane)。而數據網路(data network),則是泛指傳統網路(the Internet),即為大多數時候,手機使用者想要存取(access)的服務之所在位置。

在務實面(physically)上將系統切分成三個節點,分別模擬使用者端發送、轉發、

為了準確的區分 traffic generating 與 forwarding 的中央處理器(以下簡稱 CPU)使用率,避免單一 CPU 需要同時產生流量與處理流量,我們的測試方法將流量產生器(traffic generator)、核心網路、流量接收者 (traffic receiver) 分別部屬於不同的機器上,讓三者得以完全妥善使用三顆不同之 CPU。

\begin{figure}[b]
  \centering
  \begin{tikzpicture}

\tikzset{vertex/.style = {shape=circle,draw,minimum size=1.5em}}
\tikzset{edge/.style = {->,> = latex'}}
% vertices
\node[vertex, minimum size=1.5em] (a) at (10,) {node 1};
\node[vertex, minimum size=1.5em] (b) at (15,) {node 2};
\node[vertex, minimum size=1.5em] (c) at (20,) {node 3};
%edges
\draw[edge] (a) to (b);
\draw[edge] (b) to (c);

\end{tikzpicture}

  % [] 放的是顯示在 list of figure 的文字
  % {} 放的是顯示在圖下方的文字
  \caption[硬體架構]{{\footnotesize 硬體架構}}
  \label{fig:hardware_architecture}
\end{figure}

\section{用戶層效能分析}
\label{sec:up_evaluation}

\begin{figure}[b]
  \centering
  % 圖片的高度與寬度, height 設為 ! 代表由寬度大小等比例縮放
  \includegraphics[height=!,width=1\linewidth,keepaspectratio=true]%
  % 圖片的位置
  {figures/user_plan_performance}
  % [] 放的是顯示在 list of figure 的文字
  % {} 放的是顯示在圖下方的文字
  \caption[用戶層效能]{{\footnotesize 用戶層效能}}
  \label{fig:user_plan_performance}
\end{figure}

\section{控制層效能分析}
\label{sec:cp_evaluation}
