\chapter{背景}
\label{chapter:background}


\section{5G 核心網路簡介}
\label{sec:5g_core_intro}

% 介紹各個 NF
% 介紹 SA 與 NSA 架構差別
% 介紹 3GPP 與 non-3GPP

\subsection{5G 核心網路用戶層介紹}
\label{subsec:5g_up_intro}

% UPF 細節
% 介紹 GTP, PFCP

\subsection{5G 核心網路控制層介紹}
\label{subsec:5g_cp_intro}

% CP-NF 細節
% 介紹 SBI, NGAP, NAS, PFCP

\subsection{5G 常見流程}
\label{subsec:5g_procedure}

% UE Registration

% PDU Session Establishment

% UE Handover

% Paging

\subsection{現存 5G 核心網路專案}
\label{subsec:5g_core_project}

\section{OpenNetVM 平臺}
\label{sec:OpenNetVM}

% 其他類似平臺
% 直接用 dpdk, nff-go

\subsection{Data Plane Development Kit}
\label{subsec:DPDK}


\begin{figure}[htb]
  \centering
  % 圖片的高度與寬度, height 設為 ! 代表由寬度大小等比例縮放
  \includegraphics[height=!,width=0.6\linewidth,keepaspectratio=true]%
  % 圖片的位置
  {figures/dpdk_vs_kernel}
  % [] 放的是顯示在 list of figure 的文字
  % {} 放的是顯示在圖下方的文字
  \caption[封包處理比較:Linux 核心與 DPDK]{{\footnotesize 封包處理比較:Linux 核心與 DPDK \cite{dpdk}}}
  \label{fig:dpdk_vs_kernel}
\end{figure}

\subsection{Shared Memory}
\label{subsec:shared_memory}

\section{現行方案的挑戰}
\label{sec:challenge}
