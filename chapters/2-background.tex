% !TEX TS-program = XeLaTeX
\chapter{背景}
\label{chapter:background}


\section{5G 核心網路簡介}
\label{sec:5g_core_intro}

% 介紹各個 NF
% 介紹 SA 與 NSA 架構差別
% 介紹 3GPP 與 non-3GPP

\subsection{5G 核心網路用戶層介紹}
\label{subsec:5g_up_intro}

% UPF 細節
% 介紹 GTP, PFCP

\subsection{5G 核心網路控制層介紹}
\label{subsec:5g_cp_intro}

% CP-NF 細節
% 介紹 SBI, NGAP, NAS, PFCP

\subsection{5G 常見流程}
\label{subsec:5g_procedure}

% UE Registration
Registration是UE開啟後第一個所進行的流程,這個流程使用戶設備得以使用核心網路所帶來的服務,並且註冊並不僅僅限於用戶設備連接至核心網路的那一刻,在用戶與核心網路的連接期間,依舊會根據不同的場景執行Registration,主要分為Initial Registration,在UE開啟電源後嘗試連接至核心網路時所使用;
Periodic Registration,當UE處於CM-IDLE的狀態時,定期與核心網路回報自身存在;Mobility Registration,當UE離開Registration Area後到達新的Tracking Area後,更新UE狀態時所使用;Emergency Registration,當UE僅試圖使用核網所提供的緊急服務時所使用。由此可知,只要UE試圖與核心網路連結,必定會執行該流程,可見Registration在核心網路中的重要性。

% PDU Session Establishment
        5G核心網路所提供給UE的其中一個重要服務即是與資料網路(Data Network)的連接,資料網路並不僅限於網際網路(Internet)也包含了IP多媒體子系統(IMS)亦或是私有網路。UE為了連線至資料網路,將會發起PDU Session的建立請求,流程完成後,會建立起UE和資料網路之間的用戶平面,才能進行資料交換。
        
% UE Handover
      換手(Handover)為核心網路支援使用者移動性質的關鍵,同時確保服務的連續性。UE在行動網路供應商提供之服務範圍中移動時,由於從正在提供服務的基地台所接受訊號變差,UE為了避免服務品質低落,必須從原先提供服務之基地台切換至其他訊號較強之基地台從而維持相同服務品質,同時必須讓使用者不因切換時所造成的中斷延遲時間而感受到使用體驗上的低落。因此換手時間將直接影響了使用者體驗,並且隨著UE於長距離下的高速移動,將會大量觸發換手機制,此情形下,確保換手時的低延遲將顯得更為重要。
      
% Paging
Paging為核心網路用來尋找IDLE中的UE並且觸發訊號連接的過程,其中最常出現的場景為UE進入IDLE後,資料網路要傳送給UE的下行封包抵達核心網路,透過Paging尋找UE接著觸發Service Request建立下行PDU Session,在下行PDU Session建立前,所有的下行封包皆由UPF進行暫存,等到建立完成後一併轉發,因此低延遲表示能夠讓UE即時的接收到下行封包,也降低了UPF儲存大量下行封包下的負擔。

\subsection{現存 5G 核心網路專案}
\label{subsec:5g_core_project}

\section{OpenNetVM 平臺}
\label{sec:OpenNetVM}

% 其他類似平臺
% 直接用 dpdk, nff-go

\subsection{Data Plane Development Kit}
\label{subsec:DPDK}


\begin{figure}[htb]
  \centering
  % 圖片的高度與寬度, height 設為 ! 代表由寬度大小等比例縮放
  \includegraphics[height=!,width=0.6\linewidth,keepaspectratio=true]%
  % 圖片的位置
  {figures/dpdk_vs_kernel}
  % [] 放的是顯示在 list of figure 的文字
  % {} 放的是顯示在圖下方的文字
  \caption[封包處理比較:Linux 核心與 DPDK]{{\footnotesize 封包處理比較:Linux 核心與 DPDK \cite{dpdk}}}
  \label{fig:dpdk_vs_kernel}
\end{figure}

\subsection{Shared Memory}
\label{subsec:shared_memory}

\section{現行方案的挑戰}
\label{sec:challenge}
