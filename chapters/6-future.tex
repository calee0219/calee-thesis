\chapter{未來展望}
\label{chapter:future}


% CGO 的 overhead 很大,也許之後可以考慮用 asm 做中間層,取代 cgo
% 自動選擇 shared memory 或 DPDK
% 完成更多 SBI 使用 shared memory
% multicore LHCN
藉由本篇論文的實作與實驗可知論文的架構具備可行性,但尚有許多細節需要被改善。
該如何提供此框架更好的擴展性值得我們討論,\LHCN 基於 OpenNetVM 平臺開發,然而目前控制訊息僅能使用共享記憶體而用戶訊息則只受惠於 DPDK,是否可以使平臺自動選擇最佳化的通訊頻道?
另外像是 SBI 的最佳化,目前只完成在 SMContextCreate 上的概念驗證 (PoC),若可以完整佈署於全部 SBI 界面上,相信有助於大幅降低控制端的延遲。
而又透過傳播延遲與執行延遲的比較 (表~\ref{tab:cp_sbi_propagation}),我們知道除了通訊管道的加速外,NF 本身的運算速度亦是另一個可以大幅提升效能的地方,而像是 CGO 的使用,雖然方便快速開發,但是其效能損害卻遠比想像中的大~\cite{cgo_not_go},為加速執行效率,修改 cgo 或直接改用組語階層連接將可以是一個發展方向。
於章節~\ref{subsec:uldl_comp}中有提到,目前 \LHCN 的用戶端受限於 DPDK 的單核 CPU 輪詢,未來勢必可以考慮朝多核心的 CPU 輪詢研究,以達更佳的 CPU 使用效率。
