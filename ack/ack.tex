\begin{acknowledgement}%

    % 老師
    首先我要感謝我的指導教授陳志成教授,從我還是懵懂無知的專題生時,便已經給予我許多教導與建議。在就讀碩士班的兩年期間,老師不但提供了扎實的碩士訓練,讓我們從尋找問題、觀察現象、提出解決方法、驗證實作學習作研究的方法與態度,也會在我們遇到問到問題時不厭其煩的與我們討論並給予建議。同時,老師也提供了我們充足的實驗資源與設備,並透過與產業密切的合作,讓我們提早學習到產品開發過程中團隊開發、溝通合作、計劃訂定與實行的技巧。

    % 實驗室學長: Seb, Dobie, FuLian, 之前碩二, 之之前碩二, 謝哥, uDuck
    其次我要感謝 free5GC 的團隊夥伴,包含陳斯傑博士、邱德治工程師、翁甫廉工程師,以及參與開啟專案的學長姐們,包含奕華、冠穎、訓頡、家佐、張霽、啟恩、達魯學長,以及瑋庭學姊,在軟體開發初期,所有人一起討論問題、互相協助,才有 free5GC 現在的架構,在學長姐們的帶領下,我們才能比較快速的瞭解核心網路並進入開發狀態。尤其是特別感謝帶領團隊的陳斯傑學長,給予團隊正確的方向,在團隊毫無頭緒時提更我們方向,並且每每在我們對標準不甚理解時,也都能引導我們正確的瞭解標準內容。同時,我也想感謝張宏鉦研究員以及謝承穎學長,在我撰寫論文時給予許多建議與幫助,讓我在遇到問題時可以快速的找到解決方法並予之突破。

    % 實驗室同學: Jay, Yaowen, Syujy, Alan, David 學弟: Hao, Mao
    再來我想感謝的是實驗室的夥伴 彥傑、耀文、浚于、亮瑜、惟鈞,不管是課業上遇到問題時,或是計劃上碰到瓶頸,這群夥伴們都不離不棄的給予我協助、一同討論、一起面對問題。除了物理上的協助外,當我遇到挫折時,也願意聽我訴說苦水,給予我精神上的鼓勵。另外我也想感謝學弟妹們,尤其是浩澤學弟願意協助我在論文研究上的實驗,以及胤年學弟願意接手實驗室網管這項艱鉅的任務。

    % 家人
    最後我想感謝我的家人,能讓我毫無後顧之憂的學習、鑽研知識、進行研究,是家人給予我支持與鼓勵,才能讓我度過研究時遇到的種種壓力,讓我順利完成這篇論文。

\end{acknowledgement}
