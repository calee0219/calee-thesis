\begin{abstractzh}

    隨著行動網路演進至今,有越來越多功能需要滿足,第五代行動網路 (5G Cellular Networks) 也因應而生。5G 行動網路的三大訴求有
    \begin{enumerate*}
    \item 增強行動寬頻通訊 (eMBB)
    \item 超高可靠度低延遲通訊 (URLLC)
    \item 大規模物聯網通訊 (mMTC)
    \end{enumerate*},
    而藉由網路功能虛擬化的特性 (Network Function Virtualization, NFV),5G 核心網路可以更加方便操作、快速部屬、無痛擴展。
    然而現如今的核心網路架構是否真的能達到上訴三大訴求?還是會為網路功能虛擬化所帶來的代價而犧牲了高流量低延遲?

    本篇論文針對高流量低延遲問題,從觀察使用情境出發,提出一套節點內降低控制端 (control plane) 通訊延遲、增加用戶端 (user plane) 通訊流量的核心網路架構並予之實作。
    在不失網路功能虛擬化特性下,透過 DPDK 與 Shared memory 的操作,小心的達到訊息溝通零記憶體複製 (zero-copy)通訊,同時對於用戶端提供規則快速查找、高流量跨節點轉發功能。

    最後針對此核心網路進行效能分析,驗證可以達到超過傳統 5G 核心網路專案 11 倍的用戶端流量與低於傳統一半的控制端延遲。

\end{abstractzh}

\keywordszh{蜂巢式網路、5G、核心網路、網路功能虛擬化、高頻寬、低延遲}
